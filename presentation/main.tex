% $Header: /home/vedranm/bitbucket/beamer/solutions/conference-talks/conference-ornate-20min.en.tex,v 90e850259b8b 2007/01/28 20:48:30 tantau $

\documentclass[9pt]{beamer}

% This file is a solution template for:

% - Talk at a conference/colloquium.
% - Talk length is about 20min.
% - Style is ornate.



% Copyright 2004 by Till Tantau <tantau@users.sourceforge.net>.
%
% In principle, this file can be redistributed and/or modified under
% the terms of the GNU Public License, version 2.
%
% However, this file is supposed to be a template to be modified
% for your own needs. For this reason, if you use this file as a
% template and not specifically distribute it as part of a another
% package/program, I grant the extra permission to freely copy and
% modify this file as you see fit and even to delete this copyright
% notice. 


\mode<presentation>
{
  \usetheme{Madrid}
  % or ...

  \setbeamercovered{transparent}
  % or whatever (possibly just delete it)
}


\definecolor{light-purple}{HTML}{8900e1}
\definecolor{nyu-purple}{HTML}{57068c}
\definecolor{dark-purple}{HTML}{330662}
\definecolor{grey}{HTML}{b8b8b8}

\setbeamercolor{palette primary}{bg=nyu-purple,fg=white}
% \setbeamercolor{palette secondary}{bg=dark-purple,fg=white}
% \setbeamercolor{palette tertiary}{bg=black,fg=white}
\setbeamercolor{palette quaternary}{bg=dark-purple,fg=white} %section head left side
\setbeamercolor{structure}{fg=nyu-purple} % itemize, enumerate, etc
% \setbeamercolor{section in toc}{fg=black} % TOC sections

% Override palette coloring with secondary
\setbeamercolor{subsection in head/foot}{bg=nyu-purple,fg=white} %section head right side

\usepackage{graphicx} 
\usepackage{booktabs} 
\usepackage[english]{babel}
% or whatever

\usepackage[latin1]{inputenc}
% or whatever

\usepackage{times}
\usepackage[T1]{fontenc}
% Or whatever. Note that the encoding and the font should match. If T1
% does not look nice, try deleting the line with the fontenc.
\usepackage{textpos}

\addtobeamertemplate{frametitle}{}{% Choose your logo here
% \begin{textblock*}{100mm}(0.8\textwidth,-0.78cm)
% \includegraphics[height=0.7cm]{nyuad-logo}

% \begin{textblock*}{100mm}(0.85\textwidth,-0.75cm)
% \includegraphics[height=0.7cm]{nyu-logo}

\begin{textblock*}{100mm}(0.77\textwidth,-0.8cm)
\includegraphics[height=0.8cm]{nyush-logo}
\end{textblock*}}

\title{Unnecessarily Complicated Research Title} % Presentation title

\author[Smith, Smith, Smith]{John Smith\inst{1}, James Smith\inst{2} and Jane Smith\inst{3}} % Author(s)

\institute[NYU]{
$^1$New York University
\hspace{0.3cm}
$^2$New York University Abu Dhabi
\hspace{0.3cm}
$^3$New York University Shanghai} % Institution(s)

\date[Conference Name]% (optional, should be abbreviation of conference name)
{\today}
% - Either use conference name or its abbreviation.
% - Not really informative to the audience, more for people (including
%   yourself) who are reading the slides online

\subject{subject1, subject2}
% This is only inserted into the PDF information catalog. Can be left
% out. 



% If you have a file called "university-logo-filename.xxx", where xxx
% is a graphic format that can be processed by latex or pdflatex,
% resp., then you can add a logo as follows:

% \pgfdeclareimage[height=0.5cm]{university-logo}{university-logo-filename}
% \logo{\pgfuseimage{university-logo}}



% Delete this, if you do not want the table of contents to pop up at
% the beginning of each subsection:
\AtBeginSubsection[]
{
  \begin{frame}<beamer>{Outline}
    \tableofcontents[currentsection,currentsubsection]
  \end{frame}
}


% If you wish to uncover everything in a step-wise fashion, uncomment
% the following command: 

%\beamerdefaultoverlayspecification{<+->}


\begin{document}

\begin{frame}
  \titlepage
\end{frame}

\begin{frame}{Outline}
  \tableofcontents
  % You might wish to add the option [pausesections]
\end{frame}

\section{Introduction} 
\begin{frame}{Primary Introduction Frame}
  \begin{itemize}
    \item This is a itemize.
    \item This is a itemize.
    \item This is a itemize.
    \item This is a itemize.
    \item This is a itemize.
  \end{itemize}
\end{frame}

\begin{frame}{Plain frame}
    This is a Plain frame
\end{frame}

\section{Justification}
\begin{frame}{Justification: blocks}
  \begin{block}{Block 1}
  This is Block no.1
  \end{block}

  \begin{block}{Block 2}
  This is Block no.2
  \end{block}

\end{frame}

\section{Objectives}
\begin{frame}{Objectives}
  \begin{block}{General Objectives}
  This is a general objective block
  \end{block}

  \begin{block}{Specific Objectives}
  \begin{itemize}
      \item Specific Objectives 1
      \item Specific Objectives 2
      \item Specific Objectives 3
      \item Specific Objectives 4
  \end{itemize}
\end{block}

    
\end{frame}

\section{Fundamental Theorem}
\begin{frame}{Fundamental Theorem}
\begin{enumerate}
    \item aaa
    \item bbb
    \item ccc
\end{enumerate}
    
\end{frame}

\begin{frame}{Fundamental Theorem}
  \textbf{This} is how you make bold text.
  \medskip

  \begin{itemize}
      \item aaa
      \medskip
      \item this is how you make skips
  \end{itemize}    
  \medskip

  \begin{theorem}[Mass--energy equivalence]
    $E = mc^2$
  \end{theorem}

\end{frame}

\section{Methodology}
\begin{frame}{Methodology}
  \begin{columns}[c] % The "c" option specifies centered vertical alignment while the "t" option is used for top vertical alignment

    \column{.45\textwidth} % Left column and width
      \textbf{Methodology}
      \begin{enumerate}
        \item Statement
        \item Explanation
        \item Example
      \end{enumerate}

    \column{.5\textwidth} % Right column and width
      The second colomn

  \end{columns}
\end{frame}

\section{Results}
\begin{frame}{Resultados}
  \begin{table}
    \begin{tabular}{l l l}
      \toprule
      \textbf{Treatments} & \textbf{Response 1} & \textbf{Response 2}\\
      \midrule
      Treatment 1 & 0.0003262 & 0.562 \\
      Treatment 2 & 0.0015681 & 0.910 \\
      Treatment 3 & 0.0009271 & 0.296 \\
      \bottomrule
    \end{tabular}
    \caption{Table caption}
  \end{table}
\end{frame}

\begin{frame}{Results}
      
  \begin{figure}
      \centering
      \includegraphics[width=.7\textwidth]{placeholder}
      \caption{Place holder image}
      % \label{fig:my_label}
  \end{figure}

\end{frame}

\section{Conclusion}
\begin{frame}{Conclusion}
  \begin{itemize}
    \item more work
    \medskip
    \item more responsibility
    \medskip
    \item more satisfaction
  \end{itemize}
    
\end{frame}

\nocite{*}
\begin{frame}[allowframebreaks]{Reference}
\bibliographystyle{unsrt}
\bibliography{sample.bib}
\end{frame}
\end{document}